\section {Configuration}

%repinfo.csv

%environment variables: DBA\_TABLES

\section {Commandline tools}

\subsection{dbatbl}

\subsection{dbamsg}

\subsection{dbadb}

\section {Managing tables}

\dballe{} needs various kinds of data tables to work:

\begin{description}
\item[repinfo.csv]
  Contains informations about all the supported origins of data, including a
  text description, a short memo and the importance of the various kinds of
  data compared to the others.
\item[dballe.txt]
  Contains the description of all the possible kinds of variables that can be
  handled by \dballe{}, with information such as text description, measurement
  units, number of digits used for encoding.  It is usually referred as ``the
  local B table'', and is used for all internal representation and for
  exporting ``generic'' messages.
\item[WMO B tables]
  Data description tables maintained by WMO that are used for encoding and
  decoding BUFR and CREX messages.
\item[WMO D tables]
  Data grouping tables maintained by WMO that are used for encoding and
  decoding BUFR and CREX messages.
\end{description}

\subsection{repinfo.csv}

The file {\tt repinfo.csv} is usually installed in {\tt /etc/dballe/} and is
only read when creating or recreating a database.

It is a table encoded in CSV format, where every line is a table row and table
columns are separated by commas.  No particular string escaping is supported,
so no strings should contain a comma.

This is an example {\tt repinfo.csv}:

\begin{verbatim}
01,synop,report synottico,100,oss,0
02,metar,metar,80,oss,0
03,temp,radiosondaggio,100,oss,2
04,pilot,pilot,90,oss,2
09,boe,dati omdametrici,100,oss,31
10,ship,synop da nave,100,oss,1
11,tempship,temp da nave,100,oss,2
12,airep,airep,80,oss,4
13,amdar,amdar,100,oss,4
14,acars,acars,100,oss,4
104,ana_lm,valori analizzati LM,-1,ana,255
105,ana,analisi,-10,pre,255
106,pre_cleps_box1.5maxel001,previsti cosmo leps box 1.5 gradi valore max elemento 1,-1,pre,255
107,pre_lmn_box1.5med,previzione Lokal Model nudging box 1.5 gradi valore medio,-1,pre,255
108,pre_lmp_spnp0,previsione Lkal Model prognostica interpolato punto piu' vicino,-1,pre,255
255,generic,export generici da DB Meteo,1,oss,42
\end{verbatim}

The file has six columns:

\begin{enumerate}
\item The report code.  This is used to refer to this report type in all the
      internal representations.
\item The mnemonic short name.  It can be used to univocally refer to this report
      type in a way that is easier to remember than the report code.
\item The complete name.
\item The report priority.  When more data are found in the same physical point
      but with different report types, the report priority can be used to
      select a best value among them.
\item FIXME: Unknown: "descriptor"
\item FIXME: Unknown: "table a"
\end{enumerate}
  
\subsection{dballe.txt}
\subsection{WMO B tables}
\subsection{WMO D tables}

