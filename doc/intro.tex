\section{What is \dballe{} }

\dballe{} is a database for punctual meteorological data intended to complement
the existing tools to manage grid-based data.

These are the main characteristics of \dballe{}:

\begin{itemize}
\item it is temporary, to be used for a limited time and then be deleted.
\item does not need backup, since it only contains replicated or derived data.
\item write access is enabled for its users.
\item it is fast for both read and for write access.
\item it is based on physical principles, that is, the data it contains are
      defined in terms of omogeneous and consistent physical data.  For
      example, it is impossible for two incompatible values to exist in the
      same point in space and time.
\item it can manage fixed station and moving stations such as airplanes or
      ships.
\item it can manage both observational and forecast data.
\item it can manage data along all three dimensions in space, such as data from
      soundings and airplanes.
\item it can work based on physical parameters or on report types.
\end{itemize}


\section{Intended usage of this guide}

This guide is the reference for usage and maintenance of \dballe{}.  It
provides informations about installing a \dballe{} system and keeping it up to
the changing requirements of the users.

This guide is intended to be used by advanced users of \dballe{} that want to
know more about the system and perform some troubleshooting, and to system
administrators who need to install and maintain \dballe{} in their system.

\section{What this guide is not}

This guide does not intend to be a guide for the Fortran API: for that, the
reader is referred to the document ``Quick guide of the Fortran
API''\cite{FAPI}.

This guide does not intend to be a guide for coding with \dballe{}.  For that,
the reader is referred to the document ``libdballe
Documentation''\cite{LibDoc}.

This guide does not intend to be the reference guide for the commandline tools
that are part of \dballe{}.  For that, the reader is referred to the
manpages\cite{DbaTblMan}\cite{DbaMsgMan}\cite{DbaDbMan} of the tools.  You can
however find (see \ref{ch-tools}) an introduction to the commandline tools in
this guide, as well as some usage examples.

